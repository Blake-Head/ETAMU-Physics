\documentclass[11pt]{book}

% Use the ETAMU physics style package
\usepackage{../../shared/styles/etamu-physics}

% Document metadata
\title{Level 1 Physics\\Student Manual}
\author{ETAMU Physics Department}
\date{\today}

\begin{document}

\frontmatter
\maketitle

\tableofcontents
\listoffigures

\chapter{Introduction}

Welcome to Level 1 Physics! This manual contains problems, exercises, and reference material for the introductory physics course. 

\section{How to Use This Manual}

This manual includes:
\begin{itemize}
    \item Fundamental concepts and equations
    \item Worked examples with detailed solutions
    \item Practice problems for each topic
    \item Reference materials and formulas
\end{itemize}

\section{Mathematical Prerequisites}

You should be comfortable with:
\begin{itemize}
    \item Algebra and basic trigonometry
    \item Vector operations
    \item Basic calculus (derivatives and integrals)
\end{itemize}

\mainmatter

\chapter{Mechanics}

\section{Kinematics}

Kinematics describes motion without considering the forces that cause it. The fundamental quantities are position, velocity, and acceleration.

\subsection{Motion in One Dimension}

For motion with constant acceleration, we have three key equations:

\begin{align}
    v &= v_0 + at \label{eq:v-const-a} \\
    x &= x_0 + v_0 t + \frac{1}{2}at^2 \label{eq:x-const-a} \\
    v^2 &= v_0^2 + 2a(x - x_0) \label{eq:v2-const-a}
\end{align}

where:
\begin{itemize}
    \item $x$ is position (\si{\meter})
    \item $v$ is velocity (\si{\meter\per\second})
    \item $a$ is acceleration (\si{\meter\per\second\squared})
    \item $t$ is time (\si{\second})
\end{itemize}

\begin{examplebox}[title=Example 1.1]
A ball is thrown upward with an initial velocity of \SI{20}{\meter\per\second}. Taking upward as positive and $g = \SI{9.8}{\meter\per\second\squared}$:

\begin{enumerate}[label=(\alph*)]
    \item How long does it take to reach maximum height?
    \item What is the maximum height reached?
    \item How long is the ball in the air?
\end{enumerate}
\end{examplebox}

\textbf{Solution:}

Given: $v_0 = \SI{20}{\meter\per\second}$, $a = -g = \SI{-9.8}{\meter\per\second\squared}$

(a) At maximum height, $v = 0$. Using \cref{eq:v-const-a}:
\begin{align}
    0 &= \SI{20}{\meter\per\second} + (\SI{-9.8}{\meter\per\second\squared})t \\
    t &= \frac{\SI{20}{\meter\per\second}}{\SI{9.8}{\meter\per\second\squared}} = \SI{2.04}{\second}
\end{align}

(b) Using \cref{eq:x-const-a} with $x_0 = 0$:
\begin{align}
    x &= (\SI{20}{\meter\per\second})(\SI{2.04}{\second}) + \frac{1}{2}(\SI{-9.8}{\meter\per\second\squared})(\SI{2.04}{\second})^2 \\
    &= \SI{20.4}{\meter}
\end{align}

(c) Total time in air is twice the time to reach maximum height:
\[ t_{total} = 2 \times \SI{2.04}{\second} = \SI{4.08}{\second} \]

\subsection{Motion in Two Dimensions}

For projectile motion, we analyze horizontal and vertical components separately:

\textbf{Horizontal motion:} $a_x = 0$ (no air resistance)
\begin{align}
    x &= x_0 + v_{0x}t \\
    v_x &= v_{0x} = \text{constant}
\end{align}

\textbf{Vertical motion:} $a_y = -g$
\begin{align}
    y &= y_0 + v_{0y}t - \frac{1}{2}gt^2 \\
    v_y &= v_{0y} - gt
\end{align}

\section{Forces and Newton's Laws}

\subsection{Newton's Laws of Motion}

\begin{enumerate}
    \item \textbf{First Law:} $\sum \vect{F} = 0 \Rightarrow \vect{a} = 0$
    \item \textbf{Second Law:} $\sum \vect{F} = m\vect{a}$
    \item \textbf{Third Law:} $\vect{F}_{AB} = -\vect{F}_{BA}$
\end{enumerate}

\subsection{Common Forces}

\begin{itemize}
    \item \textbf{Weight:} $\vect{W} = m\vect{g}$ (always downward)
    \item \textbf{Normal force:} $\vect{N}$ (perpendicular to surface)
    \item \textbf{Friction:} $f_s \leq \mu_s N$ (static), $f_k = \mu_k N$ (kinetic)
    \item \textbf{Tension:} Force in strings/ropes (along the string)
\end{itemize}

\chapter{Energy and Work}

\section{Work and Kinetic Energy}

Work is defined as:
\[ W = \vect{F} \cdot \vect{d} = Fd\cos\theta \]

The work-energy theorem states:
\[ W_{net} = \Delta KE = \frac{1}{2}mv_f^2 - \frac{1}{2}mv_i^2 \]

\section{Potential Energy}

\subsection{Gravitational Potential Energy}
Near Earth's surface:
\[ U_g = mgh \]

\subsection{Elastic Potential Energy}
For a spring with spring constant $k$:
\[ U_s = \frac{1}{2}kx^2 \]

\section{Conservation of Energy}

In the absence of non-conservative forces:
\[ E = KE + PE = \text{constant} \]

\chapter{Momentum and Collisions}

\section{Linear Momentum}

Momentum is defined as:
\[ \vect{p} = m\vect{v} \]

Newton's second law in terms of momentum:
\[ \sum \vect{F} = \frac{d\vect{p}}{dt} \]

\section{Conservation of Momentum}

When no external forces act on a system:
\[ \sum \vect{p}_i = \sum \vect{p}_f \]

This applies to collisions and explosions.

\backmatter

\appendix

\chapter{Reference Formulas}

\section{Kinematics}
\begin{align}
    v &= v_0 + at \\
    x &= x_0 + v_0 t + \frac{1}{2}at^2 \\
    v^2 &= v_0^2 + 2a(x - x_0)
\end{align}

\section{Forces}
\begin{align}
    \sum \vect{F} &= m\vect{a} \\
    f_s &\leq \mu_s N \\
    f_k &= \mu_k N
\end{align}

\section{Energy}
\begin{align}
    W &= \vect{F} \cdot \vect{d} \\
    KE &= \frac{1}{2}mv^2 \\
    U_g &= mgh \\
    U_s &= \frac{1}{2}kx^2
\end{align}

\section{Momentum}
\begin{align}
    \vect{p} &= m\vect{v} \\
    \sum \vect{F} &= \frac{d\vect{p}}{dt}
\end{align}

\end{document}