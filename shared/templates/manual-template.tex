\documentclass[11pt]{book}

% Use the ETAMU physics style package
\usepackage{../../shared/styles/etamu-physics}

% Additional packages for specific needs
% \usepackage{braket}        % For quantum mechanics notation
% \usepackage{chemfig}       % For chemical formulas

% Document metadata
\title{Document Title\\Subtitle if needed}
\author{ETAMU Physics Department}
\date{\today}

\begin{document}

\frontmatter
\maketitle

\tableofcontents
\listoffigures

\chapter{Preface}

Brief introduction to the document and its purpose.

\section{How to Use This Document}

Explanation of the structure and intended use.

\mainmatter

\chapter{Chapter Title}

\section{Section Title}

Regular content goes here.

\begin{tutorialbox}[title=LaTeX Tutorial Title]
Use these boxes to explain LaTeX techniques:
\begin{verbatim}
\example{latex code here}
\end{verbatim}
This produces the desired output.
\end{tutorialbox}

\begin{examplebox}[title=Example Title]
Use these boxes for worked examples and demonstrations.
\end{examplebox}

\begin{problembox}[title=Problem Title]
Use these boxes for practice problems and exercises.
\end{problembox}

\subsection{Subsection Title}

More detailed content.

\subsubsection{Subsubsection Title}

Even more detailed content.

\chapter{Another Chapter}

Continue with additional chapters as needed.

\backmatter

\appendix

\chapter{Appendix Title}

Reference materials, formulas, constants, etc.

\end{document}